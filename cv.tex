%%%%%%%%%%%%%%%%%
% This is an example CV created using altacv.cls (v1.0.1, 11 September 2016) written by
% LianTze Lim (liantze@gmail.com), based on the 
% Cv created by BusinessInsider at http://www.businessinsider.my/a-sample-resume-for-marissa-mayer-2016-7/?r=US&IR=T
% 
%% It may be distributed and/or modified under the
%% conditions of the LaTeX Project Public License, either version 1.3
%% of this license or (at your option) any later version.
%% The latest version of this license is in
%%    http://www.latex-project.org/lppl.txt
%% and version 1.3 or later is part of all distributions of LaTeX
%% version 2003/12/01 or later.
%%%%%%%%%%%%%%%%
\documentclass[10pt,a4paper]{altacv}

%% AltaCV uses the fontawesome and academicon fonts
%% and packages. 
%% See texdoc.net/pkg/fontawecome and http://texdoc.net/pkg/academicons for full list of symbols.
%% 
%% Compile with LuaLaTeX for best results. If you
%% want to use XeLaTeX, you'll need to install
%% Academicons.ttf in your operating system's font %% folder.


% Change the page layout if you need to
%\geometry{left=1cm,right=9cm,marginparwidth=6.8cm,marginparsep=1.2cm,top=0.5cm,bottom=0.5cm}
\geometry{left=1.25cm,right=1.25cm,top=1.5cm,bottom=1.5cm,columnsep=1.2cm}


% Change the font if you want to.
\usepackage[default]{lato}
\usepackage{paracol}

% Change the colours if you want to
\definecolor{Blue}{HTML}{0c0c94} %003c69
\definecolor{VividPurple}{HTML}{3E0097}
\definecolor{SlateGrey}{HTML}{2E2E2E}
\definecolor{LightGrey}{HTML}{666666}
\definecolor{Orange}{HTML}{E36C0A}
\definecolor{Dark}{HTML}{22201f}
\colorlet{heading}{Blue}
\colorlet{accent}{Blue}
\colorlet{emphasis}{SlateGrey}
\colorlet{body}{LightGrey}
\colorlet{bold}{Dark}

% Change the bullets for itemize and rating marker
% for \cvskill if you want to
\renewcommand{\cvItemMarker}{{\small\textbullet}}
\renewcommand{\cvRatingMarker}{\faCircle}

%% sample.bib contains your publications
%\addbibresource{sample.bib}

\begin{document}
\name{Raphaël POGGI}
\tagline{Staff Embedded Software Engineer}
% Cropped to square from 
%\photo{3cm}{Gael_PORTE}
\personalinfo{%
  % Not all of these are required!
  % You can add your own with \printinfo{symbol}{detail}
  
  % Contact
  \small\makebox[0.25\linewidth][l]{\email{poggi.raph@gmail.com}}
  \small\makebox[0.25\linewidth][l]{\phone{+44 7 986 919 365}}
 \mailaddress{Camberley, GU15 3LQ, UK}
  
  % Personal info
%  \small\makebox[0.25\linewidth][l]{\wedding{Single}}
  \small\makebox[0.25\linewidth][l]{\car{Clean Driving Licence}}
  \age{33 yo}
  
  % Website
%  \small\makebox[0.25\linewidth][l]{\homepage{\href{http://portegael.com/}{Portegael.com}}}
  \small\makebox[0.25\linewidth][l]{\github{\href{https://github.com/raphui}{Github.com/raphui}}}
  \linkedin{\href{https://www.linkedin.com/in/rpoggi}{Linkedin.com/in/rpoggi}}
}

%% Make the header extend all the way to the right, if you want. Extend the right margin by 8cm (=6.8cm marginparwidth + 1.2cm marginparsep)
\makecvheader

%% Depending on your tastes, you may want to make fonts of itemize environments slightly smaller
\AtBeginEnvironment{itemize}{\small}

%% Set the left/right column width ratio to 6:4.
\columnratio{0.6}

% Start a 2-column paracol. Both the left and right columns will automatically
% break across pages if things get too long.
\begin{paracol}{2}

\cvsection{Experience}

\cvevent{Staff Embedded Software Engineer}{IDEX Biometrics}{Nov. 2017 -- Now}{Farnborough, UK/Remote}
\begin{itemize}
\item Developed and optimized \textbf{\textcolor{bold}{embedded firmware}} on \textbf{\textcolor{bold}{STM32L4}} and custom ASICs
\begin{itemize}
    \item Wrote low-level driver stack
    \item Develop a robust encrypted firmware update mechanism
    \item Worked with external partners to port the firmware on other architecture (Cortex-M33)
    \item Worked with silicon team to enable early support of the firmware on the new ASIC
    \item Implemented a fault safe storage module
    \item Developed power management functionality
\end{itemize}
\item Designed a file format to describe default parameters of the device
\begin{itemize}
    \item Designed a file format similar to JSON to describe default parameters
    \item Modified and used the device tree compiler to produce binary file from the “JSON” file
\end{itemize}
\item Developed and debugged \textbf{\textcolor{bold}{custom communications protocols}} over SPI and UART.
\begin{itemize}
    \item Developed a custom SPI protocol using 2 wires 
    \item Added support for ISO/IEC 7816-3 standard protocol
\end{itemize}
\item Implemented AES driver to enable hardware acceleration
\begin{itemize}
    \item Allowed to enable hardware acceleration of the SCP03 implementation (smartcard secure channel)
    \item Supported ECB, CBC block cipher
\end{itemize}
\item Ported and integrated SEGGER SystemView to support house-made kernel.
\item Created a cache simulator to evaluate and benchmark \textbf{\textcolor{bold}{cache architecture}} efficiency.
\begin{itemize}
    \item Supported direct and N-way associative cache, multiple cache and cache line sizes 
    \item Provided statistics of hits and misses rates
    \item Allowed to use multiple on file format
    \item Used traces from data and instruction buses
\end{itemize}
\item Ported custom ASIC hardware to \textbf{\textcolor{bold}{QEMU}}, enabling continuous integration tests without the dependency on physical hardware.
\item Collaborated daily with a distributed team (USA, UK, France, Armenia).
\end{itemize}

\divider

\cvevent{Embedded Software Engineer}{Avalun}{Feb. 2015 -- Oct. 2017}{Grenoble, France}
\begin{itemize}
\item Developed and optimized embedded firmware on STM32F429 using \textbf{\textcolor{bold}{FreeRTOS}} and custom driver stack.
\begin{itemize}
    \item Wrote the low-level drivers stack (SPI, I2C, Digital Camera Interface (DCMI), LCD, USART, TIMER, ADC, etc.)
    \item Integrated FreeRTOS
\end{itemize}
\item Designed and implemented PC control software via USB-CDC using custom script as input (C++/Qt).
\begin{itemize}
    \item Used custom script to describe the commands sent to the device
    \item Allowed the device to act like a slave and export image feeds to the PC
    \item Allowed the biomedical engineering team to develop new medical measures
\end{itemize}
\item UI development in Java using a micro JVM on top of FreeRTOS (MicroEJ – IS2T).
\item Ported and optimized signal processing algorithms from Scilab/Matlab to C
\begin{itemize}
    \item Improved processing efficiency and performance by levering ARMv7M DSP extensions
\end{itemize}
\item Enhanced datamatrix library (libdmtx) performance, significantly increasing speed.
\begin{itemize}
    \item Improved the datamatrix search algorithm speed by tailoring it to our specific case (datamatrix position and orientation in the image, size of the pattern, etc.)
    \item Reduced the memory footprint
\end{itemize}
\end{itemize}

\divider

\cvevent{Embedded Linux Engineer}{eROCCA}{Feb. 2013 -- Feb. 2015}{Grenoble, France}
\begin{itemize}
\item \textbf{\textcolor{bold}{Linux kernel drivers}} development
\begin{itemize}
    \item Custom spidev driver communicated with an 8-bit microcontroller
    \item Integrated a V4L2 camera driver to a newer Linux kernel version
    \item Improved Atmel LCB and ISI drivers to enable unsupport function (preview path)
\end{itemize}
\item Prototyped the use of LPDDR deep sleep mode to backup Linux in RAM in sleep mode and get a very fast wakeup from ultra \textbf{\textcolor{bold}{low power}} mode.
\begin{itemize}
    \item Saved processor (cpu register, MMU configuration, stack addresses) and drivers state in an external LPDDR.
    \item Power off and used an 8-bit microcontroller to trigger on a wakeup state
    \item Retrieved the saved context, power on the processor and restore its state.
    \item Very fast wakeup time and improved power consumption
\end{itemize}
\item Developed a BSP for SAMA5D3 and SAM9M10 Atmel SoC using \textbf{\textcolor{bold}{Buildroot}}, \textbf{\textcolor{bold}{U-Boot}} and \textbf{\textcolor{bold}{Linux  Kernel}}
\begin{itemize}
    \item Maintained and used buildroot to generate the filesystem
    \item Maintained U-Boot
%\begin{itemize}
%    \item Added missing support for mtd page with 8k pages (in nand\_base.c/atmel\_nand.c)
%    \item Added support for two different boards
%\end{itemize}
\end{itemize}
\item Hardware debugging using oscilloscope.
\item Designed and implemented a streaming audio and video pipeline using Gstreamer.
\end{itemize}

\divider

%\cveventsingle{Open Source Projects}{}{}{}
%\marginpar{
%\medskip
%%\vspace*{\dimexpr1pt-\baselineskip}
%\raggedright
%  \textbf{\textcolor{Orange}{\href{https://git.kernel.org/pub/scm/linux/kernel/git/torvalds/linux.git/log/?qt=author&q=poggi.raph@gmail.com}{Linux Kernel}}}
%  \begin{itemize}
%  \item Atmel nand driver improvement for 8K page size nand.
%  \item Improve error handling in TI wireless driver.
%  \end{itemize}
%}

%\iffalse
% use ONLY \newpage if you want to force a page break for
% ONLY the currentc column
\newpage

%% Switch to the right column. This will now automatically move to the second
%% page if the content is too long.
\switchcolumn

% ABOUT ME
\cvsection{About Me}
\begin{quote}
{\small As a seasoned Embedded Software Engineer with over 10 years of experience, I specialize in low-level systems programming and kernel development. My expertise includes developing and optimizing embedded firmware, designing robust firmware update mechanisms, and actively contributing to open-source projects. I have a strong track record of collaborating with distributed teams to deliver high-performance solutions. I am eager to leverage my skills to contribute to innovative technologies and support the development of secure and efficient systems.}
\end{quote}

% EDUCATION
\cvsection{Education}
\cvevent{Master's degree in Computer Science}{SUPINFO}{2013}{Grenoble, France}

% PROGRAMMING SKILLS
\cvsection{SKILLS}

\cvachievementMultiLine{\faLinux}{C, Python}{
	x86 and ARM Assembly\newline
	Linux Kernel Development\newline
	Real-Time OS Design\newline
	Embedded Firmware
}

\cvachievementMultiLine{\faCogs}{ARMv7M / ARMv8A}{SPI, I2C, I2S, UART}

\cvachievement{\faBug}{Oscilloscope, GDB, Logic Analyzer, JTAG}{}

\cvachievementMultiLine{\faUnlock}{AES (EBC, CBC, CMAC)}{
	Secure communication protocols (SCP03)\newline
	Reverse Engineering\newline
	Binary Exploitation
}

\cvachievement{\faCodeBranch}{Git, SVN, Atlassian Software suite}{}

% LANGUAGES
\cvsection{Languages}

\cvskill{French}{5}
\cvskill{English}{5}

\newpage

% HOBBIES
\cvsection{Hobbies}

\cvtag{\faMotorcycle}{Motocross}{\faCodeBranch}{Open Source}

\smallskip

\cvtag{\faLifeRing}{Sports}{\faGlobe}{Travelling}

\cvsection{Open Source}
\textbf{\textcolor{Orange}{\href{https://github.com/raphui/rnk}{rnk}}}
\begin{itemize}
\item Open source \textbf{\textcolor{bold}{real-time OS}} targeting Cortex-M4 platforms (especially STM32F4xx/STM32L4xx boards)
\item Leverage MPU and privileged/unprivileged modes
\item Support static and dynamic application, \textbf{\textcolor{bold}{device tree}} and SEGGER SystemView
\item Support device tree (memory footprint optimized) to describe hardware
\item Written from scratch
\end{itemize}

\medskip
\reversemarginpar 
\textbf{\textcolor{Orange}{\href{https://git.pengutronix.de/cgit/barebox/log/?qt=author&q=poggi.raph@gmail.com}{Barebox - Bootloader}}}
\begin{itemize}
\item Add support of \textbf{\textcolor{bold}{ARM64}} architecture
    \begin{itemize}
        \item Added the very first support of ARM64 architecture for the project
        \item Support for MMU 3 levels pagination
        \item Initial support was done on QEMU Virt machine
    \end{itemize}
\item Add support of \textbf{\textcolor{bold}{device tree}} for Atmel SoC drivers
    \begin{itemize}
        \item Added pinctrl and i2c drivers
    \end{itemize}
\item Add support of UDOO iMX6 Quad board
\end{itemize}

\medskip
\textbf{\textcolor{Orange}{\href{https://git.kernel.org/pub/scm/linux/kernel/git/torvalds/linux.git/log/?qt=author&q=poggi.raph@gmail.com}{Linux Kernel}}}
\begin{itemize}
\item Atmel nand driver improvement for 8K page size nand.
\item Improve error handling in TI wireless driver.
\end{itemize}



\end{paracol}

\end{document}
